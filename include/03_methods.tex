\section{Methods}

\subsection{Model-predictive Control}
\ac{mpc} arose from optimal control as one answer on how to \qm{close the loop} (\cite{rawlings2017mpc}).\ 
In optimal control, a system's behaviour is predicted into the future, while optimising the inputs to the system, such that a cost function is minimised.\
The working principle of \ac{mpc} is repeatedly measuring the system's state, solving an \ac{ocp}, and applying the first few of the calculated inputs.\ 
\todo{move to background or introduction}

\subsection{Wave generation}

\subsection{Adaptive weight selection}
When applying \ac{mpc}, we do not know exactly how the system will perform over the deployment of the control.\ 
In the case of the WEC-DEG when driving the system with a set weighting, different sea states will result in a different damage accumulation over time.\ 
Since the \ac{deg} has to be replaced once it breaks down, an operator of multiple \acp{deg} might be interested replacing all of the devices at once to save monetary costs.\ 
This means, that the breakdown of all the devices has to be synced, e.g. by changing the weighting of the damage cost function in a way, such that the accumulated damage cost at the designated break-down time does not exceed a fixed value $\Jd_2$.\ 

One easy way of deciding, if this damage goal is achievable with the current weighting is by evaluating the \ac{mpc} performance over a short time into the past, estimating the average rate of damage accumulation, and predicting the damage at the break-down time.\ 
\todo{can be shortened to just reflect the change of weighting}
\begin{algorithm}[htb]
\KwIn{$\Jd_i$, \td, $N$, \tf, $k \geq 2$, }
solve Problem~\ref{prob:ocp};
\end{algorithm}