\section{Introduction}
\Acp{wec} are one of the most promising techniques for  extracting energy from ocean waves.\ 
Although many different forms of \acp{wec} were studied in the last years (\todo{add citations to OWC and this snake like thing?}), their high technological complexity and deployment costs have hindered these technologies from being used in the field (\cite{Pecher2017}).\ 
One promising solution to these problems is the usage of \acp{deg}, lightweight polymeric generator with a simple physical principle and cheap production costs.\\
In a previous iteration, we investigated the behaviour of such an \ac{deg}-\ac{wec} in an optimal control setting using monochromatic waves (see \cite{Hoffmann2022moocp_wcdeg}).\ 
We showed that multi-objective optimal control gives great trade-offs between accumulated damage and extracted energy, allowing for a reduction of damage by more than 50 \% while only having to give up 1 \% of energy, assuming that we can predict the tidal motion far into the future.\\
In reality, ocean waves are irregular and therefore unpredictable for long time-horizons.\ This makes the usage of optimal control difficult, as errors in the prediction of the wave excitation result in faulty control signals.\ Under the assumption, that a correct prediction shortly into the future is possible, \ac{mpc} can be used to generate a suitable control signal during operation, having greater adaptability to environmental changes than optimal control, while still respecting target cost functions and constraints.\
The \acp{mpc} algorithm computes control inputs by solving an \acp{ocp} with shorter prediction horizon and applying the resulting control input partially.\
Therefore our contribution is the exploration of \ac{mpc} for \ac{deg}-\acp{wec} under the influence of panchromatic waves.\\
One key question our paper tries to answer is how long the \ac{mpc} prediction horizon needs to be, so that the resulting control does not deviate from an optimal solution.\ [\todo{find citation for wave sensing }]
\todo{panchromatic or stochastic. Need to stay consistent. The codebase uses stochastic so if we decide on panch. i need to refactor some stuff}
For that, we evaluate the deviation of the \ac{mpc} solution with a ground truth \ac{ocp} input signal to find that reasonable prediction horizon lengths have to include multiple wave periods.\ 
Additionally, the ever-changing conditions, under which the presented system operates, demand an adapting controller if long-time goals are to be met.\ In a real-world application it would be beneficial to track the accumulation of damage done to the membrane so a prediction of the time of failure is possible. Even better would be a way to actively control the point in time when the system breaks down. For this, we designed a simple heuristic switching scheme that can adapt the \acp{mpc}. 
This adaptation is achieved using multi-objective \ac{mpc} with a simple heuristic for changing the weights in the multi-objective optimal control scheme.
We show, that even a rudimentary switching scheme is effective in limiting the damage accumulation over an arbitrary threshold.

To the best of the authors' knowledge, this work is the first to analyse the usage of MPC for DEG-WECs.\ \cite{Faedo2017} published an extensive literature overview over the usage of MPC for general WEC-devices.\ \cite{Brekken2011} uses MPC on the example of a point absorber WEC. \todo{more examples?}