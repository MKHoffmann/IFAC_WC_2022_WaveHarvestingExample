%===============================================================================
% ifacconf.tex 2022-02-11 jpuente  
% Template for IFAC meeting papers
% Copyright (c) 2022 International Federation of Automatic Control
%===============================================================================
\documentclass[algo2e]{ifacconf}

\usepackage{graphicx}      % include this line if your document contains figures
\usepackage{natbib}        % required for bibliography

\usepackage{amsmath}
\usepackage{amsfonts}
\usepackage{amssymb}
\usepackage[nolist]{acronym}

\usepackage{listings}
\lstset{literate=%
	{Ö}{{\"O}}1
	{Ä}{{\"A}}1
	{Ü}{{\"U}}1
	{ß}{{\ss}}1
	{ü}{{\"u}}1
	{ä}{{\"a}}1
	{ö}{{\"o}}1
	{~}{{\textasciitilde}}1
}

\usepackage{multicol}
%\usepackage[font=small]{caption}
%\usepackage{subcaption}
\usepackage{enumerate}
\usepackage{graphicx}  
\usepackage{xcolor}
%\usepackage{algorithmic}
%\usepackage{algorithm}

\usepackage{siunitx}
\usepackage{xspace}

\usepackage[ruled]{algorithm2e}
%\usepackage[english]{babel}
\usepackage[utf8]{inputenc}

\usepackage{color}
\usepackage{mathtools}

%\usepackage{tikz}
%\usepackage{pgfplots}

\usepackage{multirow}

%\usepackage{gensymb} % for \degree celsius
\usepackage{eurosym} 

\usepackage{nicefrac}

\usepackage{smartref}

\let\ifacconfcaptionwidth\captionwidth
\usepackage[caption=false]{subfig}
\let\captionwidth\ifacconfcaptionwidth

%\usepackage[hidelinks]{hyperref} % to hide ugly red boxes

%%%%%%%%%%%%%%%%%%%% Functions %%%%%%%%%%%%%%%%%%%%%%%
\newcommand{\todo}[1]{\textcolor{red}{#1}}
% refs
\newcommand{\figref}[1]{Fig.~\ref{#1}}

% Technical Terms
\newcommand{\qm}[1]{``{#1}''}

% Mathematical expressions
\newcommand{\pvec}{\ensuremath{\boldsymbol{p}}}
\newcommand{\Jvek}{\ensuremath{\boldsymbol{J}}\xspace}
	\newcommand{\Ji}[1][i]{\ensuremath{J_{#1}}\xspace}
\newcommand{\cf}[1][\pvec]{\ensuremath{\Jvek(#1)}}
	\newcommand{\cfi}[2][\pvec]{\ensuremath{\Ji[#2](#1)}\xspace}

% Variables
\newcommand{\Eth}{\ensuremath{E_\mathrm{th}}\xspace}
\newcommand{\Ebd}{\ensuremath{E_\mathrm{bd}}\xspace}
\newcommand{\hl}{\ensuremath{h_\mathrm{l}}\xspace}
\newcommand{\nd}{\ensuremath{n_\mathrm{d}}\xspace}
\newcommand{\Jd}{\ensuremath{J^\mathrm{d}}\xspace}
\newcommand{\tf}{\ensuremath{t_\mathrm{f}}\xspace}
\newcommand{\td}{\ensuremath{t_\mathrm{d}}\xspace}
\newcommand{\mud}{\ensuremath{\mu_\mathrm{d}}\xspace}
\newcommand{\xr}{\ensuremath{x_\mathrm{r}}\xspace}
\newcommand{\ur}{\ensuremath{u_\mathrm{r}}\xspace}
\newcommand{\Jps}{\ensuremath{J_\mathrm{ps}}\xspace}
\newcommand{\iw}{\ensuremath{i_\mathrm{w}}\xspace}
\newcommand{\tbd}{\ensuremath{t_\mathrm{bd}}\xspace}
\newcommand{\nw}{\ensuremath{n_\mathrm{w}}\xspace}

\newcommand{\xvek}{\ensuremath{\xi}}
\newcommand{\ign}[1]{}

\newtheorem{problem}{Problem}
\newtheorem{assump}{Assumption}


% Math Operators
\DeclareMathOperator*{\minimize}{minimize}
\newcommand{\st}{\text{subject to}}

\graphicspath{{images/}}
%===============================================================================
\begin{document}
	\begin{frontmatter}
		
		\title{Multi-objective Model-Predictive Control for Dielectric Elastomer Wave Harvesters} 
		% Title, preferably not more than 10 words.
		
		\author[First]{Matthias K. Hoffmann} 
		\author[First]{Lennart Heib} 
		\author[Second]{Gianluca Rizzello}
		\author[Third]{Giacomo Moretti}
		\author[First]{Kathrin Flaßkamp} 
		
		\address[First]{Systems Modelling and Simulation,\linebreak (e-mail:~\{{matthias.hoffmann, kathrin.flasskamp\}@uni-saarland.de, lennartheib@gmail.com})}
		\address[Second]{Adaptive Polymer Systems,\linebreak	(e-mail:~gianluca.rizzello@imsl.uni-saarland.de)}
		\address[Third]{Università degli Studi di Trento, (e-mail:~giacomo.moretti@unitn.it)
			\linebreak
			$^{*}$ and $^{**}$ from Saarland University, Saarbrücken, Germany.}
		
		\begin{abstract}                % Abstract of not more than 250 words.
		\Acp{wec} are a promising technique for extracting renewable energy from ocean waves.\ 
		Extending previous work, here, we consider the multi-objective \ac{mpc} of \ac{deg}-based \acp{wec} under the influence of stochastic waves.\
		The control aims to minimise the accumulated damage of the \ac{wec} while maximising the extracted energy.\
		First, we analyse how close the MPC performance is to the ground truth optimal solution to find a good trade-off between accuracy and shortness of prediction horizon length.\ 
		As the ocean state changes, we argue that the control needs to be adapted.\
		We realize this adaptation by changing the weighting of the cost functions in a simple heuristic that aims to fulfil the long-time goal of accumulating a certain damage in a fixed time.\ 
		A simulated case-study is done to show the success of the MPC for this type of application.\
		The heuristic even outperforms some of the fixed weight MPC algorithms.
		\end{abstract}
		
		\begin{keyword}
			Multi-objective Optimal Control, Model-predictive Control, Energy Harvesting, Non-Linear Optimization, Dielectric Elastomer Generators
		\end{keyword}
		
	\end{frontmatter}
	%===============================================================================
	\begin{acronym}
	\acro{deg}[DEG]{dielectric elastomer generator}
%	\acro{awds}[AWDS]{Adaptive Weight Determination Scheme}
	\acro{moo}[MOO]{multi-objective optimization}
	\acro{ocp}[OCP]{Optimal Control Problem}
	\acro{moocp}[MOOCP]{multi-objective Optimal Control}
	\acro{wec}[WEC]{wave energy converter}
	\acro{mpc}[MPC]{Model-predictive Control}
	\acro{pop}[POP]{Pareto optimal point}
	\acro{nlp}[NLP]{non-linear program}
	\acro{foh}[FOH]{first-order hold}
\end{acronym}
	
	\section{Introduction}
\acp{wec} are one of the most promising techniques for extracting energy from ocean waves.\ 
Although many different forms of \acp{wec} were studied in the last years (\todo{add citations to OWC and this snake like thing?}), their high technological complexity and deployment costs have hindered these technologies from being used in the field.\ 
One promising solution to these problems is the usage of \acp{deg}.
 In a previous iteration, we investigated the behavior of such an \acp{deg}-\acp{wec} in an \acp{ocp} setting using monochromatic waves. [\todo{citation needed}] In reality, ocean waves are irregular and therefore unpredictable. This makes the use of \acp{ocp} difficult since an unpredictable disturbance is difficult to model with an \acp{ocp} algorithm. \todo{ask mathias about explicit ocp and write why its not applicable}. Under the assumption, that limited prediction into the future is possible, a \acp{mpc} algorithm can be used to generate a suitable control signal during operation.
The \acp{mpc} algorithm solves the \acp{ocp} by iteratively solving a smaller \acp{mpc} bla bla 
(\todo {maybe mathias can write 2 lines for the mpc algorithm here because you can probably do it better}). 
Therefore our contribution is the exploration of \ac{mpc} for \ac{deg}-\acp{wec} under the influence of panchromatic waves.\  One key question our paper tries to answer is how long does the \acp{mpc}-horizon need to be. Even though some prediction into the future is possible, the accuracy of the prediction decreases for long predictions. [\todo{find citation for wave sensing }]
\todo{panchromatic or stochastic. Need to stay consistent. The codebase uses stochastic so if we decide on panch. i need to refactor some stuff}
For that, we evaluate the deviation of the \todo {mpc closed loop solution needs to be distinguished from mpc solution} \ac{mpc} solution with a ground truth \ac{ocp} input signal to find that reasonable prediction horizon lengths have to include multiple wave periods.\ 
Additionally, the ever-changing conditions, under which the presented system operates, demand an adapting controller if long-time goals are to be met.\ In a real-world application it would be beneficial to track the accumulation of damage on the membrane (\todo replce damage on ) so a prediction of the time of failure is possible. Even better would be a way to actively control the point in time when the system breaks down. For this, we designed a simple heuristic switching scheme that can adapt the \acp{mpc}. 
This adaptation is achieved using multi-objective \ac{mpc} with a simple heuristic for changing the weights in the \ac{moocp} scheme.
We show, that even a rudimentary switching scheme is effective in limiting the damage accumulation over an arbitrary threshold.

	\section{Model and Problem Statement}
	\section{Methods}

\subsection{Discretisation and simplification of the \ac{ocp}}
In order to solve the Problem~\ref{prob:ocp}, we employ direct methods for optimal control to first formulate the \ac{ocp} as a \ac{nlp} (see \cite{Gerdts2011OC}).\ 
Using gradient-based methods, the discretised optimal control sequence is calculated.\ 
The integral terms inside the cost functions have to be approximated.\ 
We do so by adding the integrand to the dynamics
\begin{align*}
	\dot{\Upsilon}_1 &= B_h\delta^2 +z^\intercal S_r z +\dfrac{u}{R_0} - d \delta \nonumber\\
	\dot{\Upsilon}_2 &= \left(\max \{ u-E_{th}^2 h_l^2,0 \} \right)^{n_d},
\end{align*}\ 
with
\begin{align*}
	\Upsilon_1(0) = \Upsilon_2(0) = 0.
\end{align*}
Remark, that the term $\cos^2(\theta)$ is simplified to 1, the same is done in the constraint on $u$.
The extended state reads
$\xvek = \begin{bmatrix}
	{\theta} &
	{\vartheta} & 
	{z{^\intercal}} &
	{\Upsilon_1} &
	{\Upsilon_2}
\end{bmatrix}^\intercal$, with the initial value $\xvek_0 = \begin{bmatrix}
{\theta_0} &
{\vartheta_0} & 
{z_0{^\intercal}} &
{0} &
{0}
\end{bmatrix}^\intercal$

In the following, the discretised values corresponding to their continuous counterparts are marked by square brackets, e.g. the state vector $\xvek[k]$, the extended state $k$ time steps into the future.\ 
The current state of the system is advanced by one step into the future using the classical Runge-Kutta-Method of 4-th order (RK4), denoted by $F_{RK4}(\xvek[k], u[k], u[k+1], d[k], d[k+1])$.\ 
Consecutive values for the input and wave excitation are used to model \ac{foh} behaviour.\ 
The dynamics can then be expressed with the equality constraints
\begin{align*}
	\xvek[k+1] = F_{RK4}(\xvek[k], u[k], u[k+1], d[k], d[k+1]) \\
	\forall k\in\left[0,N-2]\right]\\
	\xvek[0] = \xvek_0,
\end{align*}
where $N$ is the number of time steps $t_f$ is separated into.\ 
The cost functions are then $J_1 \approx \Upsilon_1[N-1]$ and $J_2 \approx \Upsilon_2[N-1]$, so that the \ac{moocp} is
\begin{problem}\label{pb:problem_disc}
%	The constrained discrete-time minimisation problem with scalarised multi-objective cost is given by 
	\begin{align}\label{eq:problem2}
		\minimize_{u[1],\ldots,u[N]} \ &w_1 J_1 + w_2 J_2 \nonumber\\
		\ \st \ & \xvek[k+1] = F_{RK4}(\xvek[k], u[k], u[k+1],\dots\nonumber\\ &d[k]), d[k+1])  \forall \ k \in [0, N-2], \nonumber\\
		&0 \le  u[k] \le (E_{bd} h_l)^2, \ \forall \ k \in [0, N-1]\\
		& \xvek[0] = \xvek_0\\
		& u[0] = u_0,
	\end{align}
\end{problem}
where $u_0$ is a fixed initial value of 0 for the very first and $u_\mathrm{MPC}[k-1]$ for the $k$-th MPC-step. \todo{understandable?}\
The values for $\Psi$ are dropped.\ 
Since $\Psi(0)$ is a constant for the \ac{foh} formulation, it does not change the optimization problem.\ 
Regarding $\Psi(\tf)$, $I_\mathrm{h}, K_\mathrm{h}$ are multiple magnitudes larger than $C_0$, so their terms dominate the expression of $\Psi$.\ 
The quadratic cost terms push the solution to the equilibrium position $\theta = 0$ at the end of the prediction horizon, an effect unwanted in continuous operation.

\subsection{Wave generation}
\begin{figure}[htb]
	\centering
	\fontsize{8}{0}\selectfont
	\def\svgwidth{0.49\textwidth}
	\input{images/wave_excitation.pdf_tex}
	\caption{}
	\label{}
\end{figure}

\subsection{Adaptive weight selection}
When applying \ac{mpc}, we do not know exactly how the system will perform over the deployment of the control.\ 
In the case of the WEC-DEG when driving the system with a fixed weighting, different sea states will result in a different damage accumulation over time.\ 
Since the \ac{deg} has to be replaced once it breaks down, an operator of multiple \acp{deg} might be interested replacing all of the devices at the same time to save monetary costs.\ 
This means, that the breakdown of all the devices has to be synced, e.g. by changing the weighting of the damage cost function in a way, such that the accumulated damage cost at the designated break-down time \tbd does not exceed a fixed value $\Jd$.\ 

Assume a fixed set $w$ of \nw weight combinations sorted from low to high priority for the handled cost function (in our case $J_2$) and an initial weight index $\iw \in [1, \nw]$.\ 
One easy way of deciding, if the damage goal is achievable with the current weighting is by evaluating the \ac{mpc} performance over $N_p$ time steps into the past.\ 
The average rate of damage accumulation \Jps is estimated and the damage at the break-down time is predicted as an average trend.\
If the predicted damage exceeds \Jd, \iw is decreased by 1.\ 
Otherwise, if the predicted damage falls below $c_\mathrm{d}\Jd$ with $c_\mathrm{d}\in\left[0, 1\right]$, \iw is increased by 1.
This is done every $N_p$ steps if the \ac{deg} was actuated during that time.\
Of course, this algorithm just evaluates the performance in hindsight, not using the full potential of MPC.\ 
Still, we show that even this very simple heuristic can perform quite well, motivating more elaborate adaptation algorithms.
%Algorithm~\ref{alg:w_select} shows the selection, where $k$ is the current time step.
%
%\todo{is probably not necessary with the repo}
%\begin{algorithm2e}\label{alg:w_select}
%	\caption{Weight selection for $i$-th cost function}
%	\KwIn{$\Jd, \tbd, \iw, \nw, J^\mathrm{r}_i, \mud, N_p, \Delta t, k$}
%	\KwOut{\iw}
%	\If{$k+1 \; \mathrm{mod} \; N_p = 0$}{
%		$\Jps \gets \frac{1}{N_p\Delta t} \sum_{l=0}^{N_p-1} J_i^\mathrm{r}[k-l]$\;
%		\uIf{$\sum_{l=0}^{j} + \Jps(\tbd - k\Delta t) > \Jd$ and $\iw < \nw$}{
%			$\iw = \iw+1$
%		}\ElseIf{$\sum_{l=0}^{j} + \Jps(\tbd - k\Delta t) > \Jd$ and $\iw > 1$}{
%			$\iw = \iw-1$
%		}
%	}
%\end{algorithm2e}

For an exemplary implementation in MATLAB using the mentioned system, refer to \todo{link}.
	\section{Numerical Results}

The following simulations were done using MATLAB.\ 
The optimisation problems were formulated and solved using the CasADi package by \cite{Andersson2019casadi} and the IPOPT solver by \cite{Waechter2006ipopt}.\
The panchromatic waves were generated using 50 frequencies with a base frequency of \SI{0.1}{\hertz}, while perfect prediction is assumed.\ 
%\subsection{Fractional Brownian Motion Noise}

\subsection{Accuracy of model-predictive control}
\label{sec:accuracy}
When employing \ac{mpc}, we cannot expect to get the same results an \ac{ocp} with longer prediction horizon would give.\ 
By shifting the prediction horizon each time step, new information is provided, potentially leading to largely different input sequences compared to previous solutions.\
\figref{fig:mpc_error} shows the mean absolute error (MAE) between the ground truth \ac{ocp} solution over a prediction horizon of \SI{320}{\second} and $u_\mathrm{MPC}$.\ 
$u_\mathrm{MPC}$ was calculated using different horizon lengths from 10 to \SI{77}{\second}.\
As expected, the error decreases for longer prediction horizons.\ 
Moreover, it shows that for prediction horizons of length close to the period of the base wave, the error stagnates, indicating that prediction horizons longer than that are recommended.\ 
In the following, we will use a \SI{60}{\second} horizon for a good performance while not increasing the \ac{ocp}'s complexity too much.
\begin{figure}[htb]
	\centering
	\fontsize{8}{0}\selectfont
	\def\svgwidth{0.47\textwidth}
	\input{images/MPC_absolute_error.pdf_tex}
	\caption{Mean absolute deviation of $u_\mathrm{MPC}$ over \SI{320}{\second} from ground truth for different prediction horizon lengths.\ Ground truth is an OCP over the full horizon.}
	\label{fig:mpc_error}
\end{figure}

\subsection{Weight selection algorithm}
In this section, we evaluate the performance of the simple heuristic weight selection algorithm by simulating the system's behaviour for different sea states.\ 
The $\nw=15$ predetermined weights $w_2$ are evenly distributed between 0.05 and 0.95 with $w_1+w_2=1$.\ 
As shown in \figref{fig:PF}, this does not result in an even distribution of Pareto points along the Pareto front, but, as we will show, even this very simple approach yields acceptable results.\
\begin{figure}[htb]
	\centering
	\fontsize{8}{0}\selectfont
	\def\svgwidth{0.4\textwidth}
	\input{images/twoParetoFronts.pdf_tex}
	\caption{For an even distribution of weights, the Pareto points are typically not evenly distributed along the Pareto front, especially as the shape changes due to different sea states.}
	\label{fig:PF}
\end{figure}

%Alternatively, the weights could be found using the adaptive weight determination scheme by \cite{Ryu2019}.\
%\\ 
We compare for three different wave scenarios the performance of the \ac{mpc} with weight controller for two target damage values $\Jd = \left\{0.3, 0.5\right\}$.\ 
The performance of a weighting is evaluated every \SI{25}{\second}.\ 
An increase of the damage weighting is allowed after each evaluation, a decrease only every two evaluations to put more emphasis on keeping the damage low.

\begin{figure}[tb]
	\centering
	\fontsize{8}{0}\selectfont
	\def\svgwidth{0.49\textwidth}
	\input{images/mpc_paretofront_9.pdf_tex}
	\caption{The accumulated costs for the fixed weight MPC with the valid weights for the weight-controller and the extreme point approximations for the left case from \figref{fig:weight_control}.\ The costs resulting from the weight controller with threshold 0.5 dominate some of the fixed-weight costs.}
	\label{fig:mpc_pf}
\end{figure}
\begin{figure*}[htb!]
	\centering
	\fontsize{8}{0}\selectfont
	\def\svgwidth{0.97\textwidth}
	\input{images/dmg_controller_energy.pdf_tex}
	\caption{Evaluation of the \ac{mpc} with weight controller for three wave scenarios and two target damage values.\ For clarity, only every 100th value is displayed. Top: Accumulated damage over time. Middle: The selected weight index \iw over time. A lower index corresponds to a higher weighting for the damage cost. Bottom: Extracted energy over time. For the lower damage thresholds, the extracted energy reduces only by 0.09, 0.05, and \SI{0.06}{\mega\joule} for the three cases, respectively.}
	\label{fig:weight_control}
\end{figure*}

\figref{fig:weight_control} shows the performance of this heuristic weight control for the three different wave excitations by displaying the accumulated damage in the top and the selected index in the bottom plots.\ 
In the left example, the waves are all very similar and since the initial weighting would not exploit all the damage, the damage weight is steadily decreased until, for the 0.5 threshold, the index stays around 12.\
The second example shows a smooth increase into decrease of the index, so that the accumulated damage approaches the target damage without overshooting.\ 
When sea states which allow for a low damage weight are present, the controller might select weights which are too high for upcoming scenarios.\ 
This is what leads to the overshoots in the final example.\ 
Even though the controller does not react fast enough, it only violates the threshold by less than 3 \% before selecting the weighting that accumulates the least damage.\ 
Remarkably, the difference in extracted energy

To analyse why the difference in accumulated energy is so small, we compare the weight-controlled \ac{mpc} with the fixed-weight MPC.\ 
In addition to the 15 weights used in the weight-controller, we also use 0.99 and 0.01 to approximate the extreme points that minimise one of the cost functions.\ 

\figref{fig:mpc_pf} shows that the difference in extractable energy for fixed weights is less than \SI{1}{\mega\joule} or \SI{6}{\percent}.\ 
The weight-controlled MPC's performance is not only very close to a weighting of $[0.6929 \ 0.3071]$, confirming the assumption of moderate performance, it also dominates it.\ 
Thus, in this scenario, the weight controller even outperforms some of the fixed-weight MPC controls, showing that the fixed-weight MPC is not Pareto-optimal, even if the OCP-solutions are Pareto-optimal.\ 
This discrepancy is likely due to the deviations shown in section \ref{sec:accuracy}.

	\section{Conclusion}
This work analyses the usage of model-predictive control for dielectric elastomer generator-based wave energy converters under the influence of stochastic waves.\ 
Compared to previous work, MPC is considered for control, as it gives more flexibility to changes in the prediction than optimal control, while still regarding the energetic and damage cost functions.\ 
A comparison to a ground truth OCP solution shows that the MPC deviates from it considerably, if the prediction horizon does not cover multiple base periods of the acting stochastic waves.\ 
As the sea state changes over time, the meaning of the weighting of the two cost functions changes, resulting is very different...
	\bibliography{references.bib}
\end{document}
