%===============================================================================
% ifacconf.tex 2022-02-11 jpuente  
% Template for IFAC meeting papers
% Copyright (c) 2022 International Federation of Automatic Control
%===============================================================================
\documentclass[algo2e]{ifacconf}

\usepackage{graphicx}      % include this line if your document contains figures
\usepackage{natbib}        % required for bibliography

\usepackage{amsmath}
\usepackage{amsfonts}
\usepackage{amssymb}
\usepackage[nolist]{acronym}

\usepackage{listings}
\lstset{literate=%
	{Ö}{{\"O}}1
	{Ä}{{\"A}}1
	{Ü}{{\"U}}1
	{ß}{{\ss}}1
	{ü}{{\"u}}1
	{ä}{{\"a}}1
	{ö}{{\"o}}1
	{~}{{\textasciitilde}}1
}

\usepackage{multicol}
%\usepackage[font=small]{caption}
%\usepackage{subcaption}
\usepackage{enumerate}
\usepackage{graphicx}  
\usepackage{xcolor}

\usepackage{siunitx}
\usepackage{xspace}

%\usepackage{algorithm2e}
%\usepackage[english]{babel}
\usepackage[utf8]{inputenc}

\usepackage{color}
\usepackage{mathtools}

%\usepackage{tikz}
%\usepackage{pgfplots}

\usepackage{multirow}

%\usepackage{gensymb} % for \degree celsius
\usepackage{eurosym} 

\usepackage{nicefrac}

\let\ifacconfcaptionwidth\captionwidth
\usepackage[caption=false]{subfig}
\let\captionwidth\ifacconfcaptionwidth

%\usepackage[hidelinks]{hyperref} % to hide ugly red boxes

%%%%%%%%%%%%%%%%%%%% Functions %%%%%%%%%%%%%%%%%%%%%%%
\newcommand{\todo}[1]{\textcolor{red}{#1}}
% refs
\newcommand{\figref}[1]{Fig.~\ref{#1}}

% Technical Terms
\newcommand{\qm}[1]{``{#1}''}

% Mathematical expressions
\newcommand{\pvec}{\ensuremath{\boldsymbol{p}}}
\newcommand{\Jvek}{\ensuremath{\boldsymbol{J}}\xspace}
	\newcommand{\Ji}[1][i]{\ensuremath{J_{#1}}\xspace}
\newcommand{\cf}[1][\pvec]{\ensuremath{\Jvek(#1)}}
	\newcommand{\cfi}[2][\pvec]{\ensuremath{\Ji[#2](#1)}\xspace}

% Variables
\newcommand{\Eth}{\ensuremath{E_\mathrm{th}}\xspace}
\newcommand{\Ebd}{\ensuremath{E_\mathrm{bd}}\xspace}
\newcommand{\hl}{\ensuremath{h_\mathrm{l}}\xspace}
\newcommand{\nd}{\ensuremath{n_\mathrm{d}}\xspace}
\newcommand{\Jd}{\ensuremath{J^\mathrm{d}}\xspace}
\newcommand{\tf}{\ensuremath{t_\mathrm{f}}\xspace}
\newcommand{\td}{\ensuremath{t_\mathrm{d}}\xspace}
\newcommand{\mud}{\ensuremath{\mu_\mathrm{d}}\xspace}
\newcommand{\xr}{\ensuremath{x_\mathrm{r}}\xspace}
\newcommand{\ur}{\ensuremath{u_\mathrm{r}}\xspace}
\newcommand{\Jps}{\ensuremath{J_\mathrm{ps}}\xspace}
\newcommand{\iw}{\ensuremath{i_\mathrm{w}}\xspace}
\newcommand{\tbd}{\ensuremath{t_\mathrm{bd}}\xspace}
\newcommand{\nw}{\ensuremath{n_\mathrm{w}}\xspace}

\newcommand{\xvek}{\ensuremath{\xi}}
\newcommand{\ign}[1]{}

\newtheorem{problem}{Problem}
\newtheorem{assump}{Assumption}


% Math Operators
\DeclareMathOperator*{\minimize}{minimize}
\newcommand{\st}{\text{subject to}}

\graphicspath{{images/}}
%===============================================================================
\begin{document}
	\begin{frontmatter}
		
		\title{Multi-objective Model-Predictive Control for Dielectric Elastomer Wave Harvesters} 
		% Title, preferably not more than 10 words.
		
		\author[First]{Matthias K. Hoffmann} 
		\author[First]{Lennart Heib} 
		\author[Second]{Gianluca Rizzello}
		\author[Third]{Giacomo Moretti}
		\author[First]{Kathrin Flaßkamp} 
		
		\address[First]{Systems Modelling and Simulation,\linebreak (e-mail:~\{{matthias.hoffmann, kathrin.flasskamp\}@uni-saarland.de, lennartheib@gmail.com})}
		\address[Second]{Adaptive Polymer Systems,\linebreak	(e-mail:~gianluca.rizzello@imsl.uni-saarland.de)}
		\address[Third]{Università degli Studi di Trento, (e-mail:~giacomo.moretti@unitn.it)
			\linebreak
			$^{*}$ and $^{**}$ from Saarland University, Saarbrücken, Germany.}
		
		\begin{abstract}                % Abstract of not more than 250 words.
		\Acp{wec} are a promising technique for extracting renewable energy from ocean waves.\ 
		Extending previous work, here, we consider the multi-objective \ac{mpc} of \ac{deg}-based \acp{wec} under the influence of stochastic waves.\
		The control aims to minimise the accumulated damage of the \ac{wec} while maximising the extracted energy.\
		First, we analyse how close the MPC performance is to the ground truth optimal solution to find a good trade-off between accuracy and shortness of prediction horizon length.\ 
		As the ocean state changes, we argue that the control needs to be adapted.\
		We realize this adaptation by changing the weighting of the cost functions in a simple heuristic that aims to fulfil the long-time goal of accumulating a certain damage in a fixed time.\ 
		A simulated case-study is done to show the success of the MPC for this type of application.\
		The heuristic even outperforms some of the fixed weight MPC algorithms.
		\end{abstract}
		
		\begin{keyword}
			Multi-objective Optimal Control, Model-predictive Control, Energy Harvesting, Non-Linear Optimization, Dielectric Elastomer Generators
		\end{keyword}
		
	\end{frontmatter}
	%===============================================================================
	\begin{acronym}
	\acro{deg}[DEG]{dielectric elastomer generator}
%	\acro{awds}[AWDS]{Adaptive Weight Determination Scheme}
%	\acro{moo}[MOO]{multi-objective optimization}
	\acro{ocp}[OCP]{Optimal Control Problem}
	\acro{wec}[WEC]{wave energy converter}
	\acro{mpc}[MPC]{Model-predictive Control}
%	\acro{pop}[POP]{Pareto optimal point}
\end{acronym}
	
	\section{Introduction}
\acp{wec} are one of the most promising techniques for extracting energy from ocean waves.\ 
Although many different forms of \acp{wec} were studied in the last years (\todo{add citations to OWC and this snake like thing?}), their high technological complexity and deployment costs have hindered these technologies from being used in the field.\ 
One promising solution to these problems is the usage of \acp{deg}.
 In a previous iteration, we investigated the behavior of such an \acp{deg}-\acp{wec} in an \acp{ocp} setting using monochromatic waves. [\todo{citation needed}] In reality, ocean waves are irregular and therefore unpredictable. This makes the use of \acp{ocp} difficult since an unpredictable disturbance is difficult to model with an \acp{ocp} algorithm. \todo{ask mathias about explicit ocp and write why its not applicable}. Under the assumption, that limited prediction into the future is possible, a \acp{mpc} algorithm can be used to generate a suitable control signal during operation.
The \acp{mpc} algorithm solves the \acp{ocp} by iteratively solving a smaller \acp{mpc} bla bla 
(\todo {maybe mathias can write 2 lines for the mpc algorithm here because you can probably do it better}). 
Therefore our contribution is the exploration of \ac{mpc} for \ac{deg}-\acp{wec} under the influence of panchromatic waves.\  One key question our paper tries to answer is how long does the \acp{mpc}-horizon need to be. Even though some prediction into the future is possible, the accuracy of the prediction decreases for long predictions. [\todo{find citation for wave sensing }]
\todo{panchromatic or stochastic. Need to stay consistent. The codebase uses stochastic so if we decide on panch. i need to refactor some stuff}
For that, we evaluate the deviation of the \todo {mpc closed loop solution needs to be distinguished from mpc solution} \ac{mpc} solution with a ground truth \ac{ocp} input signal to find that reasonable prediction horizon lengths have to include multiple wave periods.\ 
Additionally, the ever-changing conditions, under which the presented system operates, demand an adapting controller if long-time goals are to be met.\ In a real-world application it would be beneficial to track the accumulation of damage on the membrane (\todo replce damage on ) so a prediction of the time of failure is possible. Even better would be a way to actively control the point in time when the system breaks down. For this, we designed a simple heuristic switching scheme that can adapt the \acp{mpc}. 
This adaptation is achieved using multi-objective \ac{mpc} with a simple heuristic for changing the weights in the \ac{moocp} scheme.
We show, that even a rudimentary switching scheme is effective in limiting the damage accumulation over an arbitrary threshold.

	\section{Model and Problem Statement}

\begin{figure}[htb]
	\centering
	\fontsize{9}{0}\selectfont
	\def\svgwidth{0.5\textwidth}
	\input{images/flap.pdf_tex}
	\caption{Wave surge converter: A flap mounted to the sea floor is tilted by the wave motion. It is displayed in a generic (left) and the vertical equilibrium position (right).}%Pitching flap WEC with parallelogram DEG: generic position (left); and vertical equilibrium position (right)}
\label{fig:flap}
\end{figure}

\subsection{Background}
In this work, we investigate the \ac{mpc} of the wave surge converter (see, e.g. \cite{Whittaker2012}), displayed in \figref{fig:flap}.\ 
The device is pivoted to the sea, such that incoming waves excite an oscillatory motion.\ 
This motion distorts a \ac{deg} mounted to a deformable parallelogram \cite{Moretti2014}.\ 
Through this parallelogram, the \ac{deg} applies a torque to the flap towards the equilibrium position $\theta=0$.\ 
Applying a voltage to the \ac{deg} adds an electrostatically-induced torque in the same angular direction, making the system stiffer.\ 

When a voltage is applied to the \ac{deg}, charge carrier accumulate in the \ac{deg}.\ 
By deforming the parallelogram to a smaller area, the charge density increases, such that energy can be extracted.\  
As we showed in our previous work \cite{Hoffmann2022moocp_wcdeg}, controlling the input to maximise the energy extracted from the system requires application of large voltages, damaging the \ac{deg}-material over time, and resulting in system failure after too much damage accumulated \cite{Chen2019}.\
For that reason, we also took the damage into consideration as a second control goal, resulting in lower electric fields in the \ac{deg}.

\subsection{Model}
The dynamics of the wave surge 
\begin{equation}\label{eq:dynamics}
	\begin{split}
		\begin{bmatrix}
			\dot{\theta} \\
			\dot{\delta}  \\
			\dot{z}
		\end{bmatrix}
		& = \begin{bmatrix}
			0 & 1 & 0^{1\times n} \\
			-I_h^{-1}K_h & -I_h^{-1}B_h & -I_h^{-1}C_r \\[3pt]
			0^{n\times 1} & B_r & A_r
		\end{bmatrix}    \begin{bmatrix}
			{\theta} \\
			{\delta}  \\
			{z}
		\end{bmatrix} +  \\ &  + \begin{bmatrix}
			0 \\
			I_h^{-1}  \\
			0^{n\times 1}
		\end{bmatrix} \left(d -C_0\theta u \right),
	\end{split}
\end{equation}
contain the equation of angular motion and the wave loads generated by the interaction of the waves with the flap.\ 
Here, $\delta = \dot{\theta}$ describes the flap's angular velocity and $z \in \mathbb{R}^n$ the $n$-dimensional state vector for wave radiation (\cite{Yu1995}).\
The input $u$ is the electrostatic force generated by the \ac{deg} when a voltage $v$ is applied.\ 
Since the \ac{deg}'s electrostatic force is proportional to the $v^2$, $u$ is restricted to be positive.

\subsection{Cost functions}
Our aim is the simultaneous minimisation of damage and maximisation of extracted energy, which we will employ in a \ac{moocp} setting.\ 
The extracted energy cost function is modelled as the negative generated energy
\begin{equation}\label{eq:J1}
	\begin{split}
		& J_1 = \Psi(t_f)-\Psi(0) +\int_0^{t_f} \left(B_h\dot{\theta}^2 +z^\intercal S_r z +\dfrac{u}{R_0} - d \dot{\theta}\right) \text{d}t \\
		& \text{with}\ \Psi = \dfrac{1}{2}I_h\dot{\theta}^2+\dfrac{1}{2}K_h{\theta}^2+ \dfrac{1}{2}z^\intercal Q_r z + \dfrac{1}{2} C_0\left(1 - \theta^2 \right) u,
	\end{split}
\end{equation}
with the storage function $\psi$ including kinematic, potential, electrostatic, and hydrodynamic energy contributions.\ 
Dissipations due to viscous, hydrodynamic, and electrical losses, and the power input by the incident wave are considered via the integral terms.\ 
\todo{more explanations?, see mathmod paper}

Under the assumption, that damage only starts accumulating after the applied electric field exceeds a threshold $\Eth$, the cost function can be formulated as
\begin{equation}\label{eq:J2}
	J_2 = \alpha \int_0^{t_f} \left(\max \{ \cos^2(\theta)u-E_{th}^2 h_l^2,0 \} \right)^{n_d} \text{d}t,
\end{equation}
with a normalisation factor $\alpha$ rendering $J_2$ dimensionless and an experimental parameter $n_\mathrm{d}$.\ 

	\section{Methods}

\subsection{Discretisation and simplification of the \ac{ocp}}
In order to solve the Problem~\ref{prob:ocp}, we employ direct methods for optimal control to first formulate the \ac{ocp} as a \ac{nlp} (see \cite{Gerdts2011OC}).\ 
Using gradient-based methods, the discretised optimal control sequence is calculated.\ 
The integral terms inside the cost functions have to be approximated.\ 
We do so by adding the integrand to the dynamics
\begin{align*}
	\dot{\Upsilon}_1 &= B_h\delta^2 +z^\intercal S_r z +\dfrac{u}{R_0} - d \delta \nonumber\\
	\dot{\Upsilon}_2 &= \left(\max \{ u-E_{th}^2 h_l^2,0 \} \right)^{n_d},
\end{align*}\ 
with
\begin{align*}
	\Upsilon_1(0) = \Upsilon_2(0) = 0.
\end{align*}
Remark, that the term $\cos^2(\theta)$ is simplified to 1, the same is done in the constraint on $u$.
The extended state reads
$\xvek = \begin{bmatrix}
	{\theta} &
	{\vartheta} & 
	{z{^\intercal}} &
	{\Upsilon_1} &
	{\Upsilon_2}
\end{bmatrix}^\intercal$, with the initial value $\xvek_0 = \begin{bmatrix}
{\theta_0} &
{\vartheta_0} & 
{z_0{^\intercal}} &
{0} &
{0}
\end{bmatrix}^\intercal$

In the following, the discretised values corresponding to their continuous counterparts are marked by square brackets, e.g. the state vector $\xvek[k]$, the extended state $k$ time steps into the future.\ 
The current state of the system is advanced by one step into the future using the classical Runge-Kutta-Method of 4-th order (RK4), denoted by $F_{RK4}(\xvek[k], u[k], u[k+1], d[k], d[k+1])$.\ 
Consecutive values for the input and wave excitation are used to model \ac{foh} behaviour.\ 
The dynamics can then be expressed with the equality constraints
\begin{align*}
	\xvek[k+1] = F_{RK4}(\xvek[k], u[k], u[k+1], d[k], d[k+1]) \\
	\forall k\in\left[0,N-2]\right]\\
	\xvek[0] = \xvek_0,
\end{align*}
where $N$ is the number of time steps $t_f$ is separated into.\ 
The cost functions are then $J_1 \approx \Upsilon_1[N-1]$ and $J_2 \approx \Upsilon_2[N-1]$, so that the \ac{moocp} is
\begin{problem}\label{pb:problem_disc}
%	The constrained discrete-time minimisation problem with scalarised multi-objective cost is given by 
	\begin{align}\label{eq:problem2}
		\minimize_{u[1],\ldots,u[N]} \ &w_1 J_1 + w_2 J_2 \nonumber\\
		\ \st \ & \xvek[k+1] = F_{RK4}(\xvek[k], u[k], u[k+1],\dots\nonumber\\ &d[k]), d[k+1])  \forall \ k \in [0, N-2], \nonumber\\
		&0 \le  u[k] \le (E_{bd} h_l)^2, \ \forall \ k \in [0, N-1]\\
		& \xvek[0] = \xvek_0\\
		& u[0] = u_0,
	\end{align}
\end{problem}
where $u_0$ is a fixed initial value of 0 for the very first and $u_\mathrm{MPC}[k-1]$ for the $k$-th MPC-step. \todo{understandable?}\
The values for $\Psi$ are dropped.\ 
Since $\Psi(0)$ is a constant for the \ac{foh} formulation, it does not change the optimization problem.\ 
Regarding $\Psi(\tf)$, $I_\mathrm{h}, K_\mathrm{h}$ are multiple magnitudes larger than $C_0$, so their terms dominate the expression of $\Psi$.\ 
The quadratic cost terms push the solution to the equilibrium position $\theta = 0$ at the end of the prediction horizon, an effect unwanted in continuous operation.

\subsection{Wave generation}
The stochastic wave is modeled as a superposition of $n$ scaled and shifted sine waves. The amplitude scaling aims to reconstruct the Brentschneider wave spectrum which in its general form reads
\begin{equation}
S_w(w)=A_B w^{-5} \exp \left(-B_B w^{-4}\right).
\end{equation}
The Brentschneider spectrum describes an average sea state when more specific climatic conditions are not available. 
It can be shown that when sufficiently many sine waves are used and the amplitudes of the sine waves fulfill
\begin{equation}
A_i=\sqrt{2 S_\omega\left(\omega_i\right) \Delta \omega_i}
\end{equation}
the final wave will correspond to the desired spectral distribution.
The parameters $A_b$ and $B_B$ can be modified to yield a wave with the desired overall force and frequency profile. Since for this paper, no specific sea state is emulated, the parameters have been chosen in such a way as to keep the resulting model trajectories within the modeling bounds.\\
Next, a random face-shift, $\Phi(t)$ is applied to the sine waves. 
To eradicate any periodicities that may arise, $\Phi(t)$ changes slowly over time. 
\\
Finally, the sine waves are multiplied with an excitation coefficient $\Gamma_w(w)$ to model the response of the flap to the wave. 
The disturbance of the flap can be written as 
\begin{equation}
    d(t)=\sum_{i = 1}^{n} \Gamma_{\mathrm{w}}(w_i) A_i(w_i) \sin \left(\omega_i t+\phi_i(t)\right).
\end{equation}
and \figref{fig:ExampleWave} shows the disturbance torgue generated by an example wave.
\begin{figure}[htb]
	\centering
	\fontsize{8}{0}\selectfont
	\def\svgwidth{0.49\textwidth}
	\input{images/wave_excitation.pdf_tex}
	\caption{Example of wave exitation applied to the flap}
	\label{fig:ExampleWave}
\end{figure}

\subsection{Adaptive weight selection}
When applying \ac{mpc}, we do not know exactly how the system will perform over the deployment of the control.\ 
In the case of the WEC-DEG when driving the system with a fixed weighting, different sea states will result in a different damage accumulation over time.\ 
Since the \ac{deg} has to be replaced once it breaks down, an operator of multiple \acp{deg} might be interested replacing all of the devices at the same time to save monetary costs.\ 
This means, that the breakdown of all the devices has to be synced, e.g. by changing the weighting of the damage cost function in a way, such that the accumulated damage cost at the designated break-down time \tbd does not exceed a fixed value $\Jd$.\ 

Assume a fixed set $w$ of \nw weight combinations sorted from low to high priority for the handled cost function (in our case $J_2$) and an initial weight index $\iw \in [1, \nw]$.\ 
One easy way of deciding, if the damage goal is achievable with the current weighting is by evaluating the \ac{mpc} performance over $N_p$ time steps into the past.\ 
The average rate of damage accumulation \Jps is estimated and the damage at the break-down time is predicted as an average trend.\
If the predicted damage exceeds \Jd, \iw is decreased by 1.\ 
Otherwise, if the predicted damage falls below $c_\mathrm{d}\Jd$ with $c_\mathrm{d}\in\left[0, 1\right]$, \iw is increased by 1.
This is done every $N_p$ steps if the \ac{deg} was actuated during that time.\
Of course, this algorithm just evaluates the performance in hindsight, not using the full potential of MPC.\ 
Still, we show that even this very simple heuristic can perform quite well, motivating more elaborate adaptation algorithms.
%Algorithm~\ref{alg:w_select} shows the selection, where $k$ is the current time step.
%
%\todo{is probably not necessary with the repo}
%\begin{algorithm2e}\label{alg:w_select}
%	\caption{Weight selection for $i$-th cost function}
%	\KwIn{$\Jd, \tbd, \iw, \nw, J^\mathrm{r}_i, \mud, N_p, \Delta t, k$}
%	\KwOut{\iw}
%	\If{$k+1 \; \mathrm{mod} \; N_p = 0$}{
%		$\Jps \gets \frac{1}{N_p\Delta t} \sum_{l=0}^{N_p-1} J_i^\mathrm{r}[k-l]$\;
%		\uIf{$\sum_{l=0}^{j} + \Jps(\tbd - k\Delta t) > \Jd$ and $\iw < \nw$}{
%			$\iw = \iw+1$
%		}\ElseIf{$\sum_{l=0}^{j} + \Jps(\tbd - k\Delta t) > \Jd$ and $\iw > 1$}{
%			$\iw = \iw-1$
%		}
%	}
%\end{algorithm2e}

For an exemplary implementation in MATLAB using the mentioned system, refer to \todo{link}.

	\section{Numerical Results}

The following simulations were done using MATLAB.\ 
The optimisation problems were formulated and solved using the CasADi package by \cite{Andersson2019casadi} and the Ipopt solver by \cite{Waechter2006ipopt}.\ 


\subsection{Fractional Brownian Motion Noise}

\subsection{Multi-objective Optimal Control}
Methods from \cite{Hoffmann2022moocp_wcdeg}.

\subsection{Model-predictive Control}
	\section{Conclusion}
This work analyses the usage of model-predictive control for dielectric elastomer generator-based wave energy converters under the influence of stochastic waves.\ 
Compared to previous work, MPC is considered for control, as it gives more flexibility to changes in the prediction than optimal control, while still regarding the energetic and damage cost functions.\ 
A comparison to a ground truth OCP solution shows that the MPC deviates from it considerably, if the prediction horizon does not cover multiple base periods of the acting stochastic waves.\ 
As the sea state changes over time, the meaning of the weighting of the two cost functions changes, resulting is very different...
	\bibliography{references.bib}
\end{document}
